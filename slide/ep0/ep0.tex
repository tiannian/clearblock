\documentclass{beamer}

\usepackage[UTF8,fontset=windows]{ctex}
\usepackage[orientation=landscape,size=custom,width=16,height=9,scale=0.5,debug]{beamerposter}

\title{Clear Block系列随笔 \\ Ep0 - 前言}
\author{天念 \thanks{dtiannian@gmail.com, dtiannian@aliyun.com}}
\date{\today}

\begin{document}
\begin{frame}
    \titlepage
\end{frame}

\section{简介}
\begin{frame}
    \frametitle{Clear Block是什么}
    \begin{itemize}
        \item Clear Clock是一个开源项目
        \item Clear Block是一系列视频随笔
        \item Clear Block也是我的个人笔记
        \item 这些视频讲述的是我个人对区块链的理解和想法
        \item 视频内容并未完全经过论证,不一定完全正确,有可能仅仅是猜想
        \item 视频中讲述的可能仅仅局限于理论
    \end{itemize}
\end{frame}

\begin{frame}
    \frametitle{Clear Block不是什么}
    \begin{itemize}
        \item Clear Block不是教程
        \item 观看这些视频不一定都需要编程知识,但一定需要对计算机学科有一定了解
        \item 观看这些视频不一定需要深入了解区块链,但需要一些区块链基础
        \item 本视频不会涉及具体区块链开发有关的内容
    \end{itemize}
\end{frame}

\begin{frame}
    \frametitle{除了视频,Clear Block还包括什么}
    \begin{itemize}
        \item 包括一些资料
        \item 包括一些笔记
        \item 甚至可能会有一些代码
    \end{itemize}
    上述的内容以及本系列视频,你可以在github上找到: \\
    https://github.com/tiannian/chearblock
\end{frame}


\begin{frame}[c]
    \begin{columns}[c]
        \begin{column}[c]{5cm}
            \rightline{\includegraphics[height=3cm]{../../assets/cc.png}}
        \end{column}
        \begin{column}[c]{5cm}
            \leftline{\includegraphics[height=3cm]{../../assets/by.png}}
        \end{column}
    \end{columns}

    \begin{columns}[c]
        \begin{column}[c]{5cm}
            \centerline{Except where otherwise noted, this work is licensed under}
            \centerline{\textbf{https://creativecommons.org/licenses/by/4.0}}
        \end{column}
    \end{columns}
\end{frame}

\end{document}

